\documentclass[11pt,a4paper]{article}
\advance\textwidth by 30mm
% \advance\leftmargin by -20mm
\advance\oddsidemargin by -15mm
\advance\evensidemargin by -15mm
\advance\textheight by 27.2mm
\advance\topmargin by -20mm
\sloppy

\title{SCL for CSO Programmers}
\author{Gavin Lowe}
\usepackage{scalalistings}

\newenvironment{compare}{%
  \begin{center}
    \def\arraystretch{1.2}
  \begin{tabular}{\|p{0.22\textwidth}p{0.22\textwidth}p{0.48\textwidth}\|}
  \hline SCL syntax & CSO syntax & Comments  \\  \hline}
{\\ \hline\end{tabular}\end{center}}

\def\gap{\hspace*{3mm}}

\begin{document}
\maketitle

This document describes SCL (Scala Concurrency Library) for those familiar
with CSO (Concurrent Scala Objects).

SCL is heavily influenced by Bernard Sufrin's CSO\@.  Most of the names of
classes and functions are unchanged.  However, a few changes have been made,
for example to provide a simpler interface.  

The emphasis in SCL is on pedagogy, rather than out-and-out performance.
SCL aims to provide a fairly minimal interface, essentially just enough for
the Concurrent Programming course.  In particular, SCL makes much less use of
factory methods than CSO.

The classes and top-level functions described below can be obtained by
including the line
\begin{scala}
import ox.scl._
\end{scala}
at the top of the file.

%%%%%

\subsection*{Threads}

In SCL, a |ThreadGroup| represents a collection of threads\footnote{Note: this
  is not the same as the class \SCALA{java.lang.ThreadGroup}.}.  The declaration
\SCALA{thread\{ comp \}} creates a (singleton) |ThreadGroup| that, when run,
executes |comp|.  As with CSO, threads can be combined in parallel using
\SCALA{\|\|}.

%% In SCL, the basic unit of a computation is a \emph{thread} of type |Thread|.
%% The declaration \SCALA{thread\{ comp \}} creates a computation that, when run,
%% executes |comp|.

%% A \emph{computation} of type |Computation| represents the parallel composition
%% of zero or more threads.  (|Thread| is a subtype of |Computation|.)  As with
%% CSO, computations can be combined in parallel using \SCALA{\|\|}.

\begin{compare}
\SCALA{thread\{comp\}} & \SCALA{proc\{comp\}} &  
  A thread that, when run, executes \SCALA{comp}. \\
% |Thread| & |PROC| & Type of a thread.  \\
|ThreadGroup| & |PROC| & Type of a collection of threads. \\
\SCALA{p \|\| q} & \SCALA{p \|\| q} & Parallel composition of |p| and |q|. \\
\SCALA{\|\| (ps)} & \SCALA{\|\| (ps)} & Parallel composition of the collection of
|ThreadGroup|s (or |PROC|s) |ps|.  \\
|run(p)| or |p.run| & |run(p)| or |p()| & Execute the threads |p|. \\
|fork(p)| or |p.fork| & |p.fork| & Executes |p| in a new JVM thread or threads.
\end{compare}

Exceptions are treated slightly differently from in CSO\@.  When |run| is
used, if a thread throws a non-|Stopped| exception, then it is caught, all the
other threads in the |ThreadGroup| are interrupted, and the exception is
re-thrown.  If a thread throws a |Stopped| exception, then all the other
threads in the |ThreadGroup| are allowed to terminate, at which point the
|Stopped| exception is re-thrown.  When |fork| is used, if any exception is
thrown, the program exits. 

%% that is caught and re-thrown
%% when the computation ends; if any other sort of exception is thrown, that
%% halts the program immediately.  When |fork| is used, if a thread throws any
%% exception, that halts the program immediately.

%% within a parallel composition throws an exception, the exception is printed
%% and the program exits.

SCL has the same constructs as CSO for handling |Stopped| errors.
%
\begin{compare}
\SCALA{repeat\{comp\}} & \SCALA{repeat\{comp\}} & 
 Repeatedly execute \SCALA{comp}, exiting cleanly if it throws a
 \SCALA{Stopped} exception. \\
\SCALA{repeat(guard)\{comp\}} & \SCALA{repeat(guard)\{comp\}} & 
 Repeatedly execute \SCALA{comp} while \SCALA{guard} is true, exiting cleanly
 if it throws a \SCALA{Stopped} exception. \\
\SCALA{attempt\{comp\}\{handler\}} & \SCALA{attempt\{comp\}\{handler\}} &
  Execute \SCALA{comp}, but if it throws a \SCALA{Stopped} exception, execute
  \SCALA{handler}. 
\end{compare}

%%%%%

\subsection*{Channels}

SCL makes no distinction between shared and unshared ports.  There are just
two types of channels: synchronous (|SyncChan|) and buffered (|BuffChan|).
SCL does not provide factory methods for channels, so a channel can be
constructed with, for example, |new BuffChan[A](size)|\footnote{The type
  \SCALA{A} of data for a \SCALA{BuffChan} must have an associated
  \SCALA{ClassTag}.  When \SCALA{A} is a parametric type parameter, it is
  enough to give the type bound \SCALA{A: scala.reflect.ClassTag}.  If the
  parameter~\SCALA{A} is subsequently instantiated with a concrete
  type~\SCALA{T}, it may be necessary to provide an additional argument
  \SCALA{scala.reflect.ClassTag(classOf[T])}.}.

As with CSO, a channel comprises an |InPort| and an |OutPort|.  The syntax for
standard sends and receives is unchanged.  The syntax for time-bounded sends
and receives is changed slightly (see below).  

The syntax for closing channels is mostly unchanged, except the |closeIn|
operation has been removed (it was equivalent to |close|).  |closeOut| has
been renamed to |endOfStream| (this operation normally signals the end of a
stream).  In addition, operations |isClosed| and |reopen| have been added.


\begin{compare}
|new SyncChan[A]| & 
  \raggedright |OneOne[A]()|, |OneMany[A]()|, |ManyOne[A]()|, |ManyMany[A]()|,
    |N2N[A](m,n)| &
   Creation of synchronous channel. \\
|new BuffChan[A](size)| & 
  \raggedright |OneOneBuf[A](size)|, |ManyManyBuf[A](size)|, 
  |N2NBuf[A](m,n,size)| &
   Creation of buffered channel with capacity |size|. \\
|in?()| & |in?()| & Receive from |InPort| |in|. \\
|out!x| & |out!x| & Send |x| on |OutPort| |out|. \\
|c.close|, |in.close| & |c.close|, |in.closeIn| & 
Fully close the channel. \\
|out.endOfStream| & |out.closeOut| & Close the channel for sending, normally
signalling the end of the stream. \\
|c.isClosed| & --- & Is the channel closed? \\
|c.reopen| & --- & Re-open a closed channel. 
\\
|in.receiveWithin| \gap|(millis)| & |in.readBefore(nanos)| & 
Receive from |in|, or timeout
after |millis|~ms\slash |nanos| ns, returning an |Option| value. 
\\
|out.sendWithin| \gap|(millis)(x)| &  |out.writeBefore| \gap|(nanos)(x)| &
Send |x| on |out|, or timeout
after |millis|~ms\slash |nanos| ns, returning a boolean value.
\end{compare}

%%%%%

\subsection*{Alternation}

The syntax for alternation (|alt| and |serve|) is largely unchanged.  Unlike
CSO, parentheses must \emph{not} be placed around the guard and port.  The
following example (a two-place buffer) illustrates most of the syntax.
%
\begin{scala}
  var x = -1; var empty = true
  serve(
    !empty && out =!=> {x} ==> { empty = true }
    | empty && in =?=> { v => x = v;  empty = false }
    | !empty && in =?=> { v => out!x; x = v }
  )
\end{scala}

Alternations in SCL have slightly fewer restrictions than in CSO\@.  It is
possible for a port to be shared between an alt and a non-alt.  A port may be
simultaneously enabled in several branches of the same alt (all but one
instance will be ignored). However, the following restrictions remain:
%
\begin{itemize}
\item A port may not be simultaneously enabled in two alts.  This restriction
  could be removed without too much difficulty.

\item A channel may not have both of its ports simultaneously enabled in
  alts.  Such usage could lead to a deadlock.
  %% This restriction is fairly necessary for synchronous channels, but, I think,
  %% unnecessary for buffered channels.
\end{itemize}
%
The implementation checks the above restrictions, throwing an exception if
they are broken. 

Unlike CSO, the expressions defining the ports are re-evaluated on each
iteration of a |serve|.  This means that a definition like the following has
the expected effect.
\begin{scala}
  serve( (if(flag) c1 else c2) =?=> { x => flag = !flag; ... } | ... )
\end{scala}

%% As with CSO, the expressions defining the ports are evaluated \emph{once} when
%% a |serve| is created, and not subsequently re-evaluated. 

%%%%%

\subsection*{Monitors/locks}

JVM monitors are outside CSO/SCL, so are unchanged.

In SCL, the class |Monitor| is replaced by a class |Lock|, although each
|Lock| supports the functionality of a CSO |Monitor|.  A |Lock| can be
acquired or released.  A computation can be protected by a |Lock| using
\SCALA{lock.mutex\{comp\}}: this ensures |comp| is executed under mutual
exclusion on the lock.  

Conditions can be created, associated with a |Lock|, and used as with CSO
monitors.

\begin{compare}
|new Lock| & |new Monitor| & Creation of lock or monitor. \\
|lock.acquire| & --- & Acquire the lock. \\
|lock.release| & --- & Release the lock. \\
\SCALA{lock.mutex\{comp\}} & 
\SCALA{monitor.withLock\{} \gap\SCALA{comp\}} & 
Execute |comp| under mutual exclusion on the lock/monitor. \\
|lock.newCondition| & |monitor.newCondition| & 
Obtain a new condition on the lock/monitor. \\
|cond.await| & |cond.await| & Wait for a signal on |cond|. \\
|cond.await(test)| & |cond.await(test)| & 
Wait for |test| to become true, rechecking when a signal on |cond| is
received. \\ 
|cond.signal| & |cond.signal| & Send a signal to a thread waiting on
|cond|. \\
|cond.signalAll| & |cond.signalAll| & Send a signal to each thread waiting on
|cond|. 
\end{compare}


%%%%%

\subsection*{Barrier synchronisations}

The implementation of a barrier for $n$ threads runs in time $O(\log n)$
(whereas the CSO implementation was $O(n)$).  The downside of this is that
each call to |sync| requires an identity parameter in the range $[0..n)$, with
  different threads providing different identities.    

Combining barriers, unlike with CSO, do not require the starting value for the
accumulation. 

\begin{compare}
|new Barrier(n)| & |new Barrier(n)| & 
  Creation of barrier object for |n| threads. \\
|barrier.sync(me)| & |barrier.sync| & 
  Synchronisation (by thread with identity |me|). \\
|new Combining|\-\gap|Barrier(n, f)| & |new Combining|\-\gap|Barrier(n, f, e)| &
  Creation of combining barrier for |n| threads, with combining function~|f|
  (and starting value~|e|). \\
|barrier.sync(me, x)| & |barrier.sync(x)| & 
  Synchronisation (by thread with identity~|me|) providing input~|x|. \\
|new AndBarrier(n)| & |new lock.AndBarrier(n)| & 
  Creation of conjunctive combining barrier. \\
|new OrBarrier(n)| & |new lock.OrBarrier(n)| & 
  Creation of disjunctive combining barrier.
\end{compare}

%%%%%

\subsection*{Semaphores}

Semaphores in SCL are very similar to as in CSO\@.  An exception is that the
|up| operation \emph{requires} that the semaphore is in the down state (a call
of |up| when the semaphore is already up is normally a programming error).  

\begin{compare}
|new Semaphore(isUp)| & |Semaphore(isUp)| & Creation of semaphore with state
given by |isUp|. 
\\
|new MutexSemaphore| & |MutexSemaphore()| & Creation of semaphore in the up
state, e.g.~for mutual exclusion. 
\\
|new Signalling|\-\gap|Semaphore| & |SignallingSemaphore()|  & 
Creation of semaphore in the down state, e.g.~for signalling. 
\\
|new Counting|\-\gap|Semaphore(permits)| & |Counting|\-\gap|Semaphore(permits)| &
Counting semaphore, with |permits| permits available initially. 
\\
|sem.up| & |sem.up| & Raise the semaphore. 
\\
|sem.down| & |sem.down| & Lower the semaphore.
\end{compare}

\subsection*{Linearizability testing}

The linearizability testing framework is incorporated within SCL\@.  (It was a
separate package from CSO\@.)  The interface has been slightly simplified from
previously. 

Each worker operating on the concrete datatype should have signature 
%
\begin{scala}
  def worker(me: Int, log: LinearizabilityLog[S, C]) = ...
\end{scala}
%
where |S| is the type of the sequential specification object, and |C| is the
type of the concurrent object being tested.  Each worker performs and logs
operations using commands of the form
%
\begin{scala}
  log(concOp, string, seqOp)
\end{scala}
%
where |concOp: C => A| is the operation performed on the concurrent object,
|seqOp: S => (A,S)| is the corresponding operation on the specification
object, and |string| is a |String| that describes the operation (with
different strings for semantically different operations).

The linearizability tester is created and run using commands of the form
%
\begin{scala}
  val tester = LinearizabilityTester[S,C](seqObj, concObj, p, worker _)
  assert(tester() > 0)
\end{scala}
%
where |seqObj| is the sequential specification object, |concObj| is the
concurrent object being tested, |p| is the number of workers to run, and
|worker| is as above. 
\end{document}
