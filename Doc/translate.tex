\documentclass[12pt,a4paper]{article}
\advance\textwidth by 30mm
% \advance\leftmargin by -20mm
\advance\oddsidemargin by -15mm
\advance\evensidemargin by -15mm

\title{SCL for CSO Programmers}
\author{Gavin Lowe}
\usepackage{scalalistings}

\newenvironment{compare}{%
  \begin{center}
  \begin{tabular}{\|p{0.2\textwidth}p{0.2\textwidth}p{0.50\textwidth}\|}
  \hline SCL syntax & CSO syntax & Comments  \\  \hline}
{\\ \hline\end{tabular}\end{center}}

\begin{document}
\maketitle

This document describes SCL (Scala Concurrency Library) for those familiar
with CSO (Concurrent Scala Objects).

SCL is heavily influenced by Bernard Sufrin's CSO\@.  Most of the names of
classes and functions are unchanged.  However, a few changes have been made,
for example to provide a simpler interface.

Philosophy.

SCL makes much less use of factory methods than CSO.

%%%%%

\subsection*{Computations}

In SCL, the basic unit of a computation is a \emph{thread} of type |Thread|.
The declaration \SCALA{thread\{ comp \}} creates a computation that when run
executes |comp|.

A \emph{computation} of type |Computation| represents the parallel composition
of zero or more threads.  (|Thread| is a subtype of |Computation|.)  As with
CSO, computations can be combined in parallel using \SCALA{\|\|}.

\begin{compare}
\SCALA{thread\{comp\}} & \SCALA{proc\{comp\}} &  
  A thread that, when run, executes \SCALA{comp}. \\
|Thread| & |PROC| & Type of a thread.  \\
|Computation| & |PROC| & Type of a collection of threads. \\
\SCALA{p \|\| q} & \SCALA{p \|\| q} & Parallel composition of |p| and |q|. \\
\SCALA{\|\| ps} & \SCALA{\|\| ps} & Parallel composition of the collection of
|Computation|s |ps|.  \\
|run(p)| & |run(p)| or |p()| & Execute the threads |p|. \\
|fork(p)| & |p.fork| & Executes |p| in a new JVM thread.
\end{compare}

Exceptions are treated slightly differently from in CSO\@.  When |run| is used,
if a thread throws a |Stopped| exception, then that is caught and re-thrown
when the computation ends; if any other sort of exception is thrown, that
halts the program immediately.  When |fork| is used, if a thread throws any
exception, that halts the program immediately.

%% within a parallel composition throws an exception, the exception is printed
%% and the program exits.

%%%%%

\subsection*{Channels}

SCL makes no distinction between shared and unshared ports.  There are just
two types of channels: synchronous (|SyncChan|) and buffered (|BuffChan|).
SCL does not provide factory methods for channels, so a channel can be
constructed with, for example, |new BuffChan[A](size)|\footnote{The type
  \SCALA{A} of data for a \SCALA{BuffChan} must have an associated
  \SCALA{ClassTag}.  When \SCALA{A} is a parametric type parameter, it is
  enough to give the type bound  \SCALA{A: scala.reflect.ClassTag}.}.

As with CSO, a channel comprises an |Inport| and an |OutPort|.  The syntax for
standard sends and receives is unchanged.  The syntax for time-bounded sends
and receives is changed slightly (see below).  

The syntax for closing channels
is mostly unchanged, except the |closeIn| operation has been removed (it was
equivalent to |close|).  

*** Not true!

In addition, a |reopen| operation has been added to
re-open a closed channel. 


\begin{compare}
|SyncChan[A]| & 
  \raggedright |OneOne[A]|, |OneMany[A]|, |ManyOne[A]|, |ManyMany[A]|, |N2N[A]| &
   Types of synchronous channels \\
|BuffChan[A| & 
  \raggedright |OneOneBuf[A]|, |ManyManyBuf[A]|, |N2NBuf[A]| &
   Types of buffered channels. \\
|c?()| & |c?()| & Receive from |InPort| |c|. \\
|c!x| & |c!x| & Send |x| on |OutPort| |c|. \\
|c.close| & |c.close|, |c.closeIn| & Fully close the channel. \\
|c.closeOut| & |c.closeOut| & Close the channel for sending. \\
|c.reopen| & --- & Reopen the channel. \\
|c.receiveWithin| \hspace*{3mm}|(millis)| & |c.readBefore(nanos)| & 
Receive from |c|, or timeout
after |millis|~ms\slash |nanos| ns, returning an |Option| value. \\
|c.sendWithin| \hspace*{3mm}|(millis)(x)| & 
  |c.writeBefore| \hspace*{3mm}|(nanos)(x)| &
Send |x| on |c|, or timeout
after |millis|~ms\slash |nanos| ns, returning a boolean value.
\end{compare}


%%%%%

\subsection*{Alternation}

The syntax for alternation (|alt| and |serve|) is largely unchanged.
Parentheses should not be placed around the guard and port.  The following
example (a two-place buffer) illustrates most of the syntax.
%
\begin{scala}
  var x = -1; var empty = true
  serve(
    !empty && out =!=> {x} ==> { empty = true }
    | empty && in =?=> { v => x = v;  empty = false }
    | !empty && in =?=> { v => out!x; x = v }
  )
\end{scala}

Alternations in SCL have slightly fewer restrictions than in CSO\@.  It is
possible for a port to be shared between an alt and a non-alt.  A port may be
simultaneously enabled in several branches of the same alt (all but one
instance will be ignored). However, the following restrictions remain:
%
\begin{itemize}
\item A port cannot be simultaneously enabled in two alts.  This restriction
  could be removed without too much difficulty.

\item A channel may not have both of its port used simultaneously in alts.
  This restriction is fairly necessary for synchronous channels, but, I think,
  unnecessary for buffered channels.
\end{itemize}
% 

As with CSO, the expressions defining the ports are evaluated \emph{once} when
a |serve| is created, and not subsequently re-evaluated. 

%%%%%

\subsection*{Monitors}


%%%%%

\subsection*{Other}

Barrier synchronisations

Semaphores.

Linearizability testing

\end{document}
