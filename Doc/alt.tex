\documentclass[12pt,a4paper]{article}

\usepackage{scalalistings}
\title{The implementation of {\scalashape alt} within SCL}
\author{Gavin Lowe}

\begin{document}
\maketitle

This note outlines the implementation of the |alt| construct within SCL (Scala
Concurrency Library). 

The execution of each iteration of an alt goes through several steps.
%
\begin{enumerate}
\item \textbf{Registration:} the alt registers itself with the ports of its
  branches.  If a port indicates at this point that it is ready, then the
  communication itself happens, and execution of the alt skips to the
  deregistration stage.
    
\item \textbf{Waiting:} if at least one branch was feasible, but none was
  ready, then the alt waits.
    
\item \textbf{Call-backs:} if a port becomes ready, it calls the
  |maybeReceive| function, which causes the main thread to be woken.  If a
  port is closed, it calls the |portClosed| function; if subsequently no
  branch is feasible, then the main thread is again woken, and it throws an
  |AltAbort|.
    
\item \textbf{Deregistration:} all remaining registered ports are
  deregistered.
    
\item \textbf{Completion:} The relevant code for the branch is executed,
  either the body of an |InPort|, or the continuation of an |OutPort|.
\end{enumerate}
%
These steps are explained in more detail in the following sections. 

Note that the call-backs are performed by the thread at the other end of the
channel.  All other steps are performed by the thread executing the alt.

Internally, the alt uses |synchronized| blocks to avoid races.  An exception
is that code that calls a function in a port is executed \emph{outside} a
|synchronized| block, to avoid deadlocks.  However, this code is executed
using the lock of the port, to avoid races in the port.  

If a call-back from a port happens during the registration phase, it is
blocked until the registration is over.  If an attempt to deregister a port is
made concurrently with a call-back from the same port, then the former is
blocked until the call-back is complete.  Note that after deregistration of a
port, there can be no subsequent call-back from that port.

%%%%%

\subsection*{Registration}

During registration, the alt iterates through the branches.  In the case of a
|serve|, the registration starts from the branch after the one that was
executed on the previous iteration. 

For each branch, if the guard is true and no earlier branch previously
registered the same port, the alt calls a function |registerIn| (for an
|InPort|) or |registerOut| (for an |OutPort|), so see whether the port is able
to communicate.  In each case, several responses are possible:
%
\begin{itemize}
\item That the port is closed, in which case the alt marks the branch as not
  enabled.

\item That the port is able to communicate.  In the case of an |InPort|, the
  response includes the value communicated.  In the case of an |OutPort|, the
  call includes a computation corresponding to the value passed, which the
  port evaluates before responding.  In each case, the communication logically
  occurs during the function call.  The alt records the index of the branch
  concerned.

\item That the port is not currently able to communicate.  In this case, the
  port stores a reference to the alt, together with the index of the branch
  (for subsequently identifying the branch in call-backs), and the iteration
  number (which is used only in subsequent assertions).  The alt records the
  branch as enabled.
\end{itemize}

Once it has completed registration, the alt notifies any call-backs, so they
can proceed.
%
If no branch was enabled, it throws an |AltAbort| exception.  Otherwise,
if a branch was able to communicate, the alt moves to the deregistration
phase.  Otherwise it moves to the waiting phase.

%%%%%

\subsection*{Waiting}

During the waiting phase, the main thread of the alt waits until a call-back
informs it that either a branch has communicated or all branches are disabled
(as a result of ports being closed).  In the latter case, it throws an
|AltAbort| exception.

%%%%%

\subsection*{Call-backs}

If a port that was unable to communicate at the point of registration
subsequently becomes ready, it calls the function |maybeReceive| (for an
|InPort|) or |maybeSend| (for an |OutPort|).  In each case, the branch index and
iteration number are passed.
%
The call-back is blocked until the registration phase is complete.  At that
point, if another communication has happened, the function responds negatively
to the port, and the communication doesn't happen.  Otherwise, the
communication goes ahead.  In the case of an |InPort|, the value to be
communicated is passed in the call-back, and stored within the branch.  In the
case of an |OutPort|, the value to be sent is calculated, and returned to the
port.  The main thread of the alt is woken up.

If a port that was unable to communicate at the point of registration
subsequently closes, it calls the function |portClosed|.  The call-back is
blocked until the registration phase is complete.  At that point, the branch
is marked as disabled.  If no other branch is enabled, the main thread of the
alt is woken up (to throw an |AltAbort|).

%%%%%

\subsection*{Deregistration}

The alt iterates over the enabled branches, and calls a function
|deregisterIn| (for an |InPort|) or |deregisterOut| (for an |OutPort|).  The
port removes its record of the alt.  

%%%%%

\subsection*{Completion}

In the case of an |InPort|, the value communicated is passed to the body of
the branch.  In the case of an |OutPort|, the continuation is executed. 

%%%%%

\subsection*{Restrictions}

At most one alt can register at a port at a time.  (This restriction could be
fairly easily removed: at present there's a single ``slot'' for storing the
alt.)

%% A port may not be enabled in two different branches of an alt: otherwise the
%% attempt to register the latter while the former is making a call-back would
%% lead to deadlock.  (This restriction could be fairly easily removed by
%% detecting and ignoring such repetitions.)

A synchronous channel may not have both of its ports in an alt.  Without this
restriction, the following can happen: 
\begin{enumerate}
\item Alt $a_1$ registers at the |InPort| of channel~$c_1$, and alt~$a_2$
  registers at the |InPort| of channel~$c_2$, with both unable to communicate
  immediately;

\item Alt $a_1$ tries to register at the |OutPort| of channel~$c_2$; the port
  performs a call-back to alt~$a_2$, which is blocked because $a_2$ is still
  registering.

\item Similarly, alt $a_2$ tries to register at the |OutPort| of
  channel~$c_1$; the port performs a call-back to alt~$a_1$, which is blocked
  because $a_1$ is still registering.
\end{enumerate}
%
This constitutes a deadlock. 

With a buffered channel, a similar scenario can occur if one port of each
channel is shared with another thread.  After the above step~1, suppose
threads~$t_1$ and~$t_2$ try to send on the channels; these leads to call-backs
to the two alts, which are blocked because the alts are still registering.
Then when the alts try to register at the |OutPort|s, these are blocked
because the other threads hold the locks on the channels.  This is again a
deadlock. 

At present, simultaneously using both ports of a channel in alts is prevented:
an exception will be thrown.


%% I think there is no need for a similar restriction concerning buffered
%% channels.  Using a buffered channel in the above scenario, no communication
%% would be possible at step~1 only if both channels are empty.  But in this
%% case, the communications in steps~2 and~3 would be able to proceed
%% immediately, without call-backs.  However, this usage is currently
%% prevented.

%%%%%

\subsection*{Implementation notes}

Much of the complexity of the implementation is within the implementation of
channels: these are implemented to be able to interact with alts.  Where that
implementation is common between synchronous and buffered channels, it is
implemented in one of |InPort|, |OutPort| or |Chan|, as appropriate.  

The body of a serve is passed as a computation, which is evaluated on each
iteration.  This is necessary to ensure that the expressions defining the
ports are evaluated on each iteration. 
 
\end{document}
