\documentclass[12pt,a4paper]{article}

\usepackage{scalalistings}
\title{The implementation of \SCALA{alt} within SCL}
\author{Gavin Lowe}

\begin{document}
\maketitle

This note outlines the implementation of the |alt| construct within SCL (Scala
Concurrency Library). 

The execution of each iteration of an alt goes through several steps.
%
\begin{enumerate}
\item \textbf{Registration:} the alt registers itself with the ports of its
  branches.  If a port indicates at this point that it is ready, then the
  communication itself happens, and execution of the alt skips to the
  deregistration stage.
    
\item \textbf{Waiting:} if at least one branch was feasible, but none was
  ready, then the Alt waits.
    
\item \textbf{Call-backs:} if a port becomes ready, it calls the
  |maybeReceive| function, which causes the main thread to be woken.  If a
  port is closed, it calls the |portClosed| function; if subsequently no
  branch is feasible, then the main port is again woken, and it throws an
  |AltAbort|.
    
\item \textbf{Deregistration:} all remaining registered ports are
  deregistered.
    
\item \textbf{Completion:} The relevant code for the branch is executed,
  either the body of an |InPort|, or the continuation of an |OutPort|.
\end{enumerate}

Note that the call-backs are performed by the thread at the other end of the
channel.  All other steps are performed by the thread executing the alt.

Code that involves calling a function in a port is executed \emph{outside} a
|synchronized| block, to avoid deadlocks.  However, it is executed using the
lock of the port.  If a call-back from a port happens during the registration
phase, it is blocked until the registration is over.  If an attempt to
deregister a port is made concurrently with a call-back from the same port,
then the former is blocked until the call-back is complete.  Note that after
deregistration of a port, there can be no subsequent call-back from that port.

%%%%%

\subsection*{Registration}

If a port is enabled in several different branches, all but the first is
ignored.


%%%%%

\subsection*{Restrictions}

At most one alt can register at a port at a time.  (This restriction could be
fairly easily removed: at present there's a single ``slot'' for storing the
alt.)

%% A port may not be enabled in two different branches of an alt: otherwise the
%% attempt to register the latter while the former is making a call-back would
%% lead to deadlock.  (This restriction could be fairly easily removed by
%% detecting and ignoring such repetitions.)

A synchronous channel may not have both of its ports in an alt.  Without this
restriction, the following can happen: 
\begin{itemize}
\item Alt $a_1$ registers at the |InPort| of channel~$c_1$, and alt~$a_2$
  registers at the |InPort| of channel~$c_2$;

\item Alt $a_1$ tries to register at the |OutPort| of channel~$c_2$; the port
  performs a call-back to alt~$a_2$, which is blocked because $a_2$ is still
  registering.

\item Similarly, alt $a_2$ tries to register at the |OutPort| of
  channel~$c_1$; the port performs a call-back to alt~$a_1$, which is blocked
  because $a_1$ is still registering.
\end{itemize}
%
This constitutes a deadlock. 

I think there is no need for a similar restriction concerning buffered
channels, but it is currently prevented. 
